% !TeX encoding = UTF-8
% !TeX root = ROSCostMapZh-Cn.tex
% !TeX TS-program = xelatex.exe

% for distanguishing different author emails
%\makeatletter
%\newcommand{\ssymbol}[1]{\textsuperscript{\@fnsymbol{#1}}}
%\makeatother

% math packages
\usepackage{amsmath, amssymb, amsfonts, mathtools}%fouriernc amsthm, 
\usepackage{bm, fixmath}% bold math
\newcommand\defeq{\stackrel{\mathclap{\normalfont\mbox{def}}}{=}}

% figures
\usepackage{graphicx}
\graphicspath{{figures/}}
\usepackage{csquotes}

% set toc depth
\setcounter{tocdepth}{3}
\setcounter{secnumdepth}{3}

% highlight texts
\usepackage{soul}
\definecolor{lightblue}{rgb}{.90, .95, 1}
%\sethlcolor{rosepale}
\sethlcolor{lime}

%% symbols: degree
\usepackage{siunitx}
%% customrized label
\usepackage{enumitem}

\usepackage[toc]{multitoc}
\renewcommand*{\multicolumntoc}{2}
\setlength{\columnseprule}{0.5pt}

\usepackage{xcolor}%[dvipsnames]
% new color
\definecolor{psychedelicpurple}{rgb}{0.87, 0.0, 1.0}
\definecolor{formalshade}{rgb}{0.95,0.95,1}
\definecolor{colorbg}{rgb}{0.92, 0.92, 0.92}
\definecolor{colorbg2}{rgb}{0.95, 0.95, 0.95}
\definecolor{darkviolet}{rgb}{0.58, 0.0, 0.83}
\definecolor{Cyan}{rgb}{0.0, 1.0, 1.0}
\definecolor{ForestGreen}{rgb}{0.13, 0.55, 0.13}
\definecolor{BurntOrange}{rgb}{0.8, 0.33, 0.0}

\definecolor{delim}{RGB}{20,105,176}
\colorlet{punct}{red!60!black}
\definecolor{orange-red}{rgb}{1.0, 0.27, 0.0}
\definecolor{portlandorange}{rgb}{1.0, 0.35, 0.21}
\renewcommand{\theFancyVerbLine}{\sffamily
	\textcolor{red}{\scriptsize
		\oldstylenums{\arabic{FancyVerbLine}}}}
	
% environment
% for adjustwidth environment
\usepackage[strict]{changepage}
\newenvironment{formal}
{%
	\def\FrameCommand
	{%
		%\hspace{1pt}%
		{\color{psychedelicpurple}\vrule width 2pt}%
		{\color{formalshade}\vrule width 4pt}%
		\colorbox{formalshade}%
	}%
	\vspace{-15pt}
	\MakeFramed{\advance\hsize-\width\FrameRestore}%
	%\noindent\hspace{-4.55pt}% disable indenting first paragraph
	\begin{adjustwidth}{}{7pt}%
		\vspace{2pt}\vspace{2pt}%
		%\indent
	}
	{%
		\vspace{2pt}
	\end{adjustwidth}
	\endMakeFramed%
}
\newcommand{\parEnCn}[2]{
	\par{\indent #1}
	\begin{formal}
		\par{~~~~#2}
	\end{formal}
	%\vspace{-10pt}
}
\newcommand{\parEnCnNoIndent}[2]{
	\noindent#1
	\begin{formal}
		\noindent#2
	\end{formal}
	%\vspace{-10pt}
}

% bib style
\usepackage[backend=biber,bibstyle=gb7714-2015,%nature,%%加载biblatex宏包,使用参考文献
citestyle=gb7714-2015,backref=true%%其中后端backend使用biber
]{biblatex}%标注(引用)样式citestyle,著录样式bibstyle都采用gb7714-2015样式

\usepackage{cleveref}
\crefname{equation}{式}{式}
\Crefname{equation}{式}{式}
\crefname{table}{表}{表}
\Crefname{table}{表}{表}
\crefname{figure}{图}{图}
\Crefname{figure}{图}{图}
\crefname{inputPrg}{程序}{程序}

\definecolor{Melon}{rgb}{0.99, 0.74, 0.71}
\definecolor{ForestGreen}{rgb}{0.13, 0.55, 0.13}
\newtcolorbox{EnvNote}[1][]{
	width=\textwidth,
	fonttitle=\bfseries,
	breakable,
	fonttitle=\bfseries\color{Brown},
	colframe=Melon,
	colback=Melon!10
	#1}
\newcommand{\TextNote}[2]{
	\begin{EnvNote}
		{\colorbox{ForestGreen}{\color{white}\kaishu 注}~} \textbf{#1}: #2
	\end{EnvNote}
}

\newcommand{\TextNoteNoT}[1]{
	\begin{EnvNote}
		{\colorbox{ForestGreen}{\color{white}\kaishu 注}~} {#1}
	\end{EnvNote}
}

\newcommand{\TextNotePlain}[1]{
	{\colorbox{ForestGreen}{\color{white}\kaishu 注}~} {#1}
}

\setminted[c++]{%gobble=0, %Remove the first n characters from each input line
	%escapeinside=||,
	baselinestretch = 1.0,
	bgcolor=colorbg2,
	breaklines, breakanywhere,
	%showspaces,
	showtabs,
	autogobble,
	mathescape,  
	%	fontsize=\small, 
	linenos,
	stepnumber = 1,
	%numbersep=5pt, 
	tabsize=4,%4
	frame=leftline, %{lines},%lines, %{leftline,topline},%{leftline,topline,bottomline},
	rulecolor=\color{BurntOrange},%\color{green},%\color{LightRubineRed}, ForestGreen
	framerule=1pt,
	%highlightcolor, %default: LightCyan
	%highlightlines={1, 5-6},
	%label=main.cpp,
	%labelposition=all
	obeytabs=false,
	mathescape=false
	samepage=false,
	showspaces=false,
	showtabs =false,
	texcl=false,
}
\setmintedinline[c++]{
	breaklines,
	breakanywhere,
	bgcolor=lightblue, %colorbgdark,%showspaces,
	autogobble,
	%fontsize=\normalsize,
	%frame=none,
	%framerule=0pt,
}

%\newtcblisting{bashcode}{
%	listing engine=minted,
%	colback=colorbg,
%	colframe=black!70,
%	listing only,
%	minted style=colorful,
%	minted language=bash,
%	minted options={linenos=true,texcl=true},
%	left=1mm,	
%}

\setminted[bash]{
	%	fontsize=\small, 
	baselinestretch = 1.1,
	bgcolor=colorbg2,
	breaklines, breakanywhere,
	breaksymbolleft={},
	%	breakanywheresymbolpre={},
	fontfamily=tt,
	linenos=true,
	numberblanklines=true,
	numbersep=5pt,
	autogobble,
	frame=leftline,
	rulecolor=\color{darkviolet},
	framesep=2mm,
	framerule=2pt,
	tabsize=4,
	obeytabs=false,
	mathescape=false
	samepage=false,
	showspaces=false,
	showtabs =false,
	texcl=false,
	breaklines=true,
}

%\definecolor{aquamarine}{rgb}{0.5, 1.0, 0.83}
%\definecolor{beaublue}{rgb}{0.74, 0.83, 0.9}
%\definecolor{springbud}{rgb}{0.65, 0.99, 0.0}% good, but repeat with hl-color
%\definecolor{purplex11}{rgb}{0.63, 0.36, 0.94}
\definecolor{magenta}{rgb}{1.0, 0.0, 1.0}
\setmintedinline[bash]{
	breaklines,
	breakanywhere,
	bgcolor=colorbg,%magenta!10, %colorbgdark,%showspaces,
	autogobble,
	%	fontsize=\normalsize,
	%frame=none,
	%framerule=0pt,
}

\setminted[xml]{bgcolor=colorbg, 
	%	fontsize=\small, 
	linenos, 
	xleftmargin=8pt,
	breaklines=true,
	frame={leftline},
	rulecolor=\color{darkviolet},
	framesep=2mm,
	framerule=2pt,
	%breaksymbolleft={},
}
\setmintedinline[xml]{
	bgcolor=colorbg, 
	%fontsize=\small, 
	bgcolor=lightblue,
	autogobble,
}

\setminted[python]{%gobble=0, %Remove the first n characters from each input line
	%escapeinside=||,
	baselinestretch = 1.0,
	bgcolor=colorbg,
	breaklines, breakanywhere,
	breaksymbolleft={},
	%showspaces,
	showtabs,
	autogobble,
	mathescape,  
	%fontsize=\normalsize, 
	linenos,
	stepnumber = 1,
	%numbersep=5pt, 
	tabsize=4,%4
	frame={lines},%lines, %{leftline,topline},%{leftline,topline,bottomline},
	rulecolor=\color{darkviolet},%\color{green},%\color{LightRubineRed}, ForestGreen
	framerule=1pt,
	%highlightcolor, %default: LightCyan
	%highlightlines={1, 5-6},
	%label=main.cpp,
	%labelposition=all
}
\setmintedinline[python]{
	bgcolor=colorbg, 
	%fontsize=\normalsize, 
	bgcolor=lightblue,
	autogobble,
}

\setminted[lua]{%gobble=0, %Remove the first n characters from each input line
	%escapeinside=||,
	baselinestretch = 1.1,
	bgcolor=colorbg2,
	breaklines, 
	breakanywhere,
	breaksymbolleft={},
	%showspaces,
	%showtabs,
	autogobble,
	mathescape,  
	%fontsize=\normalsize, 
	linenos,
	stepnumber = 1,
	%numbersep=5pt, 
	tabsize=4,%4
	frame=lines,%{leftline,topline,bottomline},
	rulecolor=\color{Orange},%\color{green},%\color{LightRubineRed}, ForestGreen
	framerule=2pt,
	%highlightcolor, %default: LightCyan
	%highlightlines={1, 5-6},
	%label=main.cpp,
	%labelposition=all
}

\setminted[xml]{%gobble=0, %Remove the first n characters from each input line
	%escapeinside=||,
	baselinestretch = 1.1,
	bgcolor=colorbg2,
	breaklines, 
	breakanywhere,
	breaksymbolleft={},
	%showspaces,
	%showtabs,
	autogobble,
	mathescape,  
	%fontsize=\normalsize, 
	linenos,
	stepnumber = 1,
	%numbersep=5pt, 
	tabsize=4,%4
	frame=lines,%{leftline,topline,bottomline},
	rulecolor=\color{ForestGreen},%\color{green},%\color{LightRubineRed}, Cyan 
	framerule=2pt,
	mathescape=false,
	%highlightcolor, %default: LightCyan
	%highlightlines={1, 5-6},
	%label=main.cpp,
	%labelposition=all
}


\setminted[matlab]{%gobble=0, %Remove the first n characters from each input line
	%escapeinside=||,
	baselinestretch = 1.0,
	bgcolor=colorbg2,
	breaklines, breakanywhere,
	%showspaces,
	showtabs,
	autogobble,  
	%	fontsize=\small, 
	linenos,
	stepnumber = 1,
	%numbersep=5pt, 
	tabsize=2,%4
	frame=leftline, %{lines},%lines, %{leftline,topline},%{leftline,topline,bottomline},
	rulecolor=\color{ForestGreen},%\color{green},%\color{BurntOrange}, ForestGreen
	framerule=1pt,
	%highlightcolor, %default: LightCyan
	%highlightlines={1, 5-6},
	%label=main.cpp,
	%labelposition=all
	obeytabs=false,
	mathescape=false
	samepage=false,
	showspaces=false,
	showtabs =false,
	texcl=false,
}


\lstdefinelanguage{json}{
	basicstyle=\normalfont\ttfamily,
	numbers=left,
	numberstyle=\scriptsize,
	stepnumber=1,
	numbersep=8pt,
	showstringspaces=false,
	breaklines=true,
	frame=lines,
	backgroundcolor=\color{colorbg},
	literate=
	*{:}{{{\color{punct}{:}}}}{1}
	{,}{{{\color{punct}{,}}}}{1}
	{\{}{{{\color{delim}{\{}}}}{1}
	{\}}{{{\color{delim}{\}}}}}{1}
	{[}{{{\color{delim}{[}}}}{1}
	{]}{{{\color{delim}{]}}}}{1},
}


%%============= for highlight Chinese texts =============%% 
% http://latexcolor.com/
\definecolor{neongreen}{rgb}{0.22, 0.88, 0.08}

\newcommand*{\cjkhl}[2]{{\def\xcjklhcolorbox{\colorbox{#1}}\xcjkhl#2\relax}}
\newcommand{\CJKNhl}[1]{
	\cjkhl{lime}{#1}%neongreen
}

\makeatletter

%look ahhead for the next character
\def\xcjkhl{\futurelet\tmp\xxcjkhl}

%helper macro to make leaders with a highlight box.
\def\xcjkhlleaders{\leavevmode\leaders\hbox{%
		\fboxsep\z@\xcjklhcolorbox{\strut\kern.1pt%
			\ifx\cjkhlbleeda\relax\else\kern\cjkhlbleeda\fi\relax}%
		\ifx\cjkhlbleeda\relax\else\kern-\cjkhlbleeda\fi\relax}}

% leaders with a 1pt of stretch/shrink to put before or after punctuation
\def\@@yhlstretch{\leavevmode\xcjkhlleaders\hskip\z@\@plus.1em \@minus.1em }

\def\@chkhlpar#1\fi\fi#2{\par\noindent\xcjkhl}

\def\@cjkhl@beforeafter#1{%
	\ifx\tmp#1%
	\@@yhlstretch
	\let\@chkhlstretch\@@yhlstretch
	\let\cjkhlpenalty\@highpenalty
	\fi}

\def\@cjkhl@after#1{%
	\ifx\tmp#1%
	\let\@chkhlstretch\@@yhlstretch
	\let\cjkhlpenalty\@highpenalty
	\fi}

\def\@cjkhl@before#1{%
	\ifx\tmp#1%
	\@@yhlstretch
	\let\cjkhlpenalty\@highpenalty
	\fi}

\def\xxcjkhl{%
	%look for a \par (from \obeylines)
	\ifx\tmp\par\expandafter\@chkhlpar\fi
	%look for a \relax to finish
	\ifx\tmp\relax
	\else
	%by default do no stretch leaders after the character
	\let\@chkhlstretch\relax
	\let\cjkhlpenalty\z@
	%these stretch before and after
	\@cjkhl@beforeafter?%
	\@cjkhl@beforeafter;%
	% these stretch after
	\@cjkhl@after,%
	\@cjkhl@after。%
	\@cjkhl@after)%
	\@cjkhl@after》%
	\@cjkhl@after”%
	%these stretch before
	\@cjkhl@before(%
	\@cjkhl@before《%
	\@cjkhl@before“%
	% look for a space
	\ifx\tmp\@sptoken
	\xxxcjkhlsp
	\else
	%default case stick the current character in a box
	\xxxcjkhl
	\fi\fi}

%make a highlight leaders stretch/shrink as much as a normal word space.
\def\xxxcjkhlsp#1\fi\fi#2{%
	\fi\fi
	\xcjkhlleaders\hskip \fontdimen2\font  plus \fontdimen3\font minus \fontdimen4\font\relax
	\xcjkhl#2}

%get out of a double \if test
\def\xxxcjkhl\fi\fi{%
	\fi\fi
	\@chkhl}

%The simple case box the current character and start looking for the next.
% bleed slightly on the right to avoid gaps showing
\let\cjkhlbleeda\relax
\def\cjkhlbleeda{.07pt}
\def\@chkhl#1{{%
		\fboxsep\z@
		\leavevmode\penalty\cjkhlpenalty
		\xcjklhcolorbox{%
			\strut#1\ifx\cjkhlbleeda\relax\else\kern\cjkhlbleeda\fi}}%
	\ifx\cjkhlbleeda\relax\else\kern-\cjkhlbleeda\fi\relax
	\@chkhlstretch
	\xcjkhl}%

\makeatother
%%============= for highlight Chinese texts =============%% 